% -----------------------------------------------------------------------
% --- DOCUMENTS ---
% -----------------------------------------------------------------------
\documentclass[11pt, a4paper, french]{article}


\usepackage[utf8]{inputenc}
\usepackage[T1]{fontenc}
\usepackage{ae, pslatex}
\usepackage[french]{babel}
\selectlanguage{french} 


\usepackage{fontspec}
\setmainfont{Liberation Serif}

\usepackage{mathtools}
\usepackage{amssymb}
\usepackage{pgfplots}
\usepackage{caption}

\usepackage{titlesec}
\usepackage{color}
\usepackage{colortbl}

\usepackage{hhline,tabu}

% -----------------------------------------------------------------------
% --- MARGES ---
% -----------------------------------------------------------------------
\usepackage{vmargin}
\setpapersize{A4}
\setmarginsrb{60pt}{50pt}{60pt}{25pt}{15pt}{25pt}{15pt}{25pt}

% -----------------------------------------------------------------------
% --- EN-TETE ET PIED DE PAGE ---
% -----------------------------------------------------------------------
\usepackage{fancyhdr}
\usepackage{lastpage}
\pagestyle{fancy}

\fancyhead[L]{SYM - Systèmes mobiles}
\fancyhead[R]{IL - TIC - HEIG-VD \\ Automne 2017}
\fancyfoot[C]{\thepage{}}

\title{Systèmes mobiles \\ Laboratoire n°1 : Introduction aux activités Android}
\author{Mathieu Monteverde, Sathiya Kirushnapillai, Zucca Michela}
\date{Automne 2017}

\titlespacing\section{0pt}{12pt plus 4pt minus 2pt}{0pt plus 2pt minus 2pt}
\titlespacing\subsection{0pt}{12pt plus 4pt minus 2pt}{0pt plus 2pt minus 2pt}
\titlespacing\subsubsection{0pt}{12pt plus 4pt minus 2pt}{0pt plus 2pt minus 2pt}

% ***********************************************************************
% *** DOCUMENT PRINCIPAL ***
% ***********************************************************************
\begin{document}

	\maketitle
    
    \setlength{\parskip}{1em}
	
	\section*{Introduction}	

    \section*{Questions}

        Comment organiser les textes pour obtenir une application multi-langues (français, allemand,italien, langue par défaut : anglais)? Que se passe-t-il si une traduction est manquante? \par
        
        Android permet de gérer les textes et leurs traductions associées à l'aide de fichier \textit{strings.xml}. Pour chaque traduction, il suffit de créer un fichier \textit{strings.xml} et de le placer dans le dossier values-*. Par exemple, values-es pour l'espagnole.\par

        Cette syntaxe correspond à la syntaxe créée par Android Studio en utilisant l'outil de traduction de langue. 
        
        La langue par défaut se trouve dans le dossier values. C'est cette dernière qui est utilisée si la traduction est manquante.

    
	

\end{document}

